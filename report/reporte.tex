\documentclass{article}
\usepackage[left=2cm,right=2cm,top=3cm,bottom=3cm,letterpaper]{geometry}
\usepackage[spanish]{babel}
\usepackage[utf8]{inputenc}

\usepackage{verbatim, array}
\usepackage{hyperref}
\usepackage{amsmath, amsfonts, amssymb}
\usepackage{graphicx}
\usepackage[T1]{fontenc}

\newcommand{\jimage}[3]{\begin{figure}[h!]\includegraphics[width=#1\textwidth]{#2}\caption{#3}\end{figure}\vskip10pt}
\newcommand{\jcimage}[3]{\begin{figure}[h!]\centering\includegraphics[width=#1\textwidth]{#2}\caption{#3}\end{figure}\vskip10pt}

\author{Héctor Enrique Gómez Morales}
\title{
  Analisis y mejoras en el procesamiento de imagenes obtenidas por Resonancia Magnetica Funcional: Resting State, Contraste BOLD. }
\date{27 de Marzo de 2015}
\begin{document}
\maketitle
\section{Introducción}

\section{Resonancia Magnetica Funcional}

\section{Contraste BOLD}

\section{Resting State}

\section{Objetivos}

\section{Entornos de Desarrollo y Procesamiento}

\section{FSL}

\section{AFNI}

\section{FCON1000}

\section{CPAC}

\section{MELODIC}

\section{Conclusiones}

\end{document}
